\problemname{Peragrams}

\illustration{.25}{peragrams.jpg}{Photo by
  \href{http://www.flickr.com/photos/crosswordman/3966825101/}{Ross Beresford}}

Per recently learned about \emph{palindromes}. Now he wants to tell us about it and also has more
awesome scientific news to share with us.

``A palindrome is a word that is the same no matter whether you read it backward or forward'', Per recently
said in an interview. He continued: ``For example, \emph{add} is not a palindrome, because reading
it backwards gives \emph{dda} and it's actually not the same thing, you see. However, if we reorder
the letters of the word, we can actually get a palindrome. Hence, we say that \emph{add} is a
\emph{Peragram}, because it is an anagram of a palindrome''.

Per gives us a more formal definition of \emph{Peragrams}: ``Like I said, if a word is an anagram of
at least one palindrome, we call it a \emph{Peragram}. And recall that an anagram of a word $w$ contains
exactly the same letters as $w$, possibly in a different order.''

\section*{Task}
Given a string, find the minimum number of letters you have to remove from it, so that the string
becomes a Peragram.

\section*{Input}
Input consists of a string on a single line. The string will contain at least $1$ and at most
$1\,000$ characters. The string will only contain lowercase letters \texttt{a-z}.

\section*{Output}
Output should consist of a single integer on a single line, the minimum number of characters that
have to be removed from the string to make it a Peragram.
